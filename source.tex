%	The document class command HAS to be the first thing in a LaTeX file. It takes
%	the following format:
%	\documentclass[options]{class}

%	The chosen options tell LaTeX to create a document using the Report class
%	which uses a4 paper, set to two sided and the font size shall be 11pt
\documentclass[11pt, a4paper, twoside]{report}

%	Some available document classes are:
%	article	- for articles in journals, presentations, short reports, program
%			  documentation, etc.
%	IEEEtran	- for articles in the IEEE transactions format.
%	proc		- for proceedings (based on article)
%	minimal	- only used for debugging purposes
%	report	- for full reports, thesis documents, small books, etc.
%	book		- for full books.
%	slides	- for slides (uses big sans serif characters)
%	memoir	- for writing full memoir-type books
%	letter		- for writing letters
%	beamer	- for writing presentations

%	User packages can be loaded in the preamble only. This means that all
%	packages that are required for the successful compilation of the document
%	need to be declared at the very root of the document files.
%	They go in here in the
%	following format:
%	\usepackage[options]{package}

%	Including the package "epigraph" allows quotes to be added at the
%	beginning of each chapter or section.
\usepackage{epigraph}

%	The following two usepackage statements MUST be included for the
%	Japanese text found in chapterOne.tex to compile correctly.
\usepackage{CJKutf8}
\usepackage[overlap]{ruby}

%	The following package is required for adding source code listings to a
%	document.
%	At the moment, there is a source code listing in the file chapterTwo.tex
\usepackage{listings}

%	The following package is used for formatting urls in the body and
%	providing a clickable link in the rendered file.
\usepackage{url}

%	The following package is used to parse mathematical characters
%	(used in preface.tex)
\usepackage{mathtools}

%	The above area is called the "preamble"

%	The \begin{document} command informs LaTeX to begin processing
%	the document proper, and that it should no longer receive any
%	\usepackage{} statements
\begin{document}

%	\title, and \author are compulsory.
%	If \date is omitted, LaTeX will put today's date in it's place. The entire
%	section has to be followed by \maketitle, otherwise it will not be
%	formatted correctly.

\title{A Selection of Examples of \LaTeX Code with Discussions}
\author {Jamie Taylor}
\maketitle

%	The abstract, if required, goes in here. It can be renamed from
%	'Abstract' to anything else by using the following command
%	(before the \begin{abstract}, obviously):
%	\renewcommand{\abstractname}{The New Name}

\begin{abstract}

%	The abstract (along with any other document content) can be
%	loaded from another file, using the following command:
%	\input{filename}

%	The above command will add the contents of the file 'filename.tex'
%	into the current document without adding any page breaks.
%	It will look use non-absolute file addressing to find the target file
%	(i.e it will look in the same directory as the current source file)

%	However, the following command will add the contents of a given
%	file between two page breaks:
%	\include{filename}

%	\include{} adds the contents of the file in place, whereas \insert{}
%	will add a page break in place, then add the contents of the file,
%	then add another page break before the content after the command.

%	I am going to include an abstract from a separate file. That way,
%	maintaining the layout of the main document will be easier.
Any \LaTeX author can tell from the word go, that the designer of \TeX (which is what \LaTeX is based on) chose a modular paradigm for this system. This choice was deliberate on the designer's part for the simple reason of allowing the author to spend more time actually writing their document rather than editting the layout and look-and-feel of the document.\\
\indent{}As a side note, the original designer (or architect, if you will) of \TeX is Donald Knuth - a champion of excellent algorithm and software design.This author belives that this fact shows when looking at the source code and layout of files for \TeX projects.\\
\indent{}The purpose of this document is to show some of the features that \LaTeX provides. These features will be hand picked by the author as the features that most appeal to him. That is not to say that the author is only interested in a particular subest of features. The author holds the view that when learning a system (beit a language, a toolset or anything else) it is better to focus on the parts of that sytsem which appeal to the learner at first. Then, once the foundation is laid, the learner can then go on to learn other parts of the system.\\
\indent{}The document will be structured such that explanations of the chosen features will be included in the body of text. Further descriptions on the correct usage of the features can be found in the document's soource code.
\begin{center}\emph{N.B: The source code for this document can be obtained from the following website:} \url{http://www.taylorj.org.uk/documents/LaTeX}\end{center}

%a couple of comments about the contents of this file:
% \\ means put a line break here
% \indent{] means indent this paragraph
% \emph{..} means italicise the text ".."

%the code used for justification (the N.B in the above abstract), is adapted from
%the code found at the following url:
%http://www.artofproblemsolving.com/Wiki/index.php/LaTeX:Layout

%	The following command tells LaTeX that we would like to end the
%	section or command (in this case the abstract)
\end{abstract}

%	Before my table of contents, I want to add a preface. It will be
%	pulled in from the file preface.tex, much like the Abstract was.
\chapter*{Preface}
%the above line ensures that the contents of this file are made into a chapter
% (called "Preface"), but that chapter is not added to the table of contents
%(that's what the * is for)

All of the text for the different chapters in this document has been pulled in from files (abstract.tex, preface.tex, chapterOne.tex, etc.) that are seperate to the main file (source.tex).\\
\indent{}This is exceptionally useful, as it means that the author can format the original \LaTeX{} document then keep all of the text that will be added to each section/chapter for said document in seperate files. In other words, if the author wants to change the content of a single section or chapter, then the he  need not worry about the impact on the design for the rest of the document.\\
\indent{}A note on the pronunciation of \TeX and \LaTeX:\\
\indent{}\TeX was originally written, by Donald Knuth, to be used by all writers alike regardless of whether those writers where compiling a technical document, a letter, a manuscript for a book (fact or fiction) or a reference guide for some system or other.\\
\indent{}To reflect this, the characters used to spell \TeX are not those found in the Latin cahracter set, but the Greek character set. Therefore: the word \TeX is spelt with the following Greek letters:\begin{math}\tau, \epsilon and \chi \end{math}(lower cse Tau, Epslion and Chi, repsectively).\\
\indent{}The pronounciation, therefore is based on the Greek alphabet. This means that \TeX is pronounced similar to the (informal) English word "tech" (whcih is short for "Technology").\\
\indent{}Similarly, the name \LaTeX is based on \TeX. This means that it is pronounced either "Lay-Tech" or "Lah-Tech". There is no hard-and-fast rule for the correct  pronouncination of \LaTeX (other than the correct pronounciation of the second part, \TeX).

%the Greek character code and how to add them was adapted from the code found
%at the wollowing url:
%https://en.wikibooks.org/wiki/LaTeX/Mathematics

%the \chapter*{..} code is adapted from an answer to a tex.stackexchange
%question. The answer can be found at the following url:
%http://tex.stackexchange.com/a/19808

%	The command informs LaTeX to add a table of contents directly
%	where command appears.
\tableofcontents

%	A list of tables and a list of figures can be added in the same way
%	as the table of contents. The following two commands are used
%	to add them.
%	\listoffigures
%	\listoftables

%	The rest of the file can be split into either chapters or sections.
%	Chapter declarations take the following format:
% 	\chapter{Name of Chapter}

%	Whereas section declarations take the following format:
%	\section{Name of Section}

%	Sections, parts and chapters are all added to the table of contents
%	(ig there is one) automatically.
%	They are also numbered automatically. But to add one that doesn't
%	appear in the table of contents and is not numbered, the following
%	command is used:
%	\section*{Name of Unreferenced, Non-Numbered Section}
%	The same can be achieved for chapters by using the following:
%	\chapter*{Name of Unreferenced Chapter}

%	Subsections and Subsubsections can be added in the same way:
%	\subsection{Name of Sub-Section}
%	\subsubsection{Name of Sub-Sub-Section}

%	The first chapter, as with the abstract and preface, will be pulled in
%	from another file. I will still use input, however. This is because the
%	\chapter command uses \cleardoublepage and \clearpage before
%	adding a new chapter. So, there is no need to use \include, just yet.

\chapter{A Beginning}
\epigraph{A journey of a thousand miles begins with a single step}{Laozi}


%a couple of comments about the contents of this file:
% \chapter{Name of chapter}	- creates a new chapter, on a new page, with the name
%					  "Name of Chapter"
%\ epigrpah{A Quote}{Author}	- adds a quote to the text
%at the minute, the epigraph at the top of the file has been commented out, as it requires
%the epigraph package. This isn't a huge proble, except that it requires all 3gigs of texLive
% be installed (but this will give access to all of the other useful packages)

%the following section of Japanese text is adapted from the tex.stackexchange found at
%the following url: http://tex.stackexchange.com/a/15528

\section{Example of Usage}
\begin{CJK}{UTF8}{min}
これは \ruby{日}{に} \ruby{本}{ほん}\ruby{語}{ご}です。\\
%\\furigana text example: \ruby{山}{やま}\ruby{田}{だ}と\ruby{上}{うえ}\ruby{田}{だ}
\end{CJK}
\indent{}One of the (many) great things about writing documents in \LaTeX is the support for packages. The above text, when added to a word processor, would produce an output \emph{without} Ruby text\footnote{see: \url{http://en.wikipedia.org/wiki/Ruby_text} for more details on Ruby text}. However, due to the modular nature of \LaTeX, adding a single reference to the package (\emph{\{ruby\}}) will add the Ruby text (sometimes called \emph{Furigana}) to the document.\\
\indent{}The following is an example of what the above Japanese characters might look like when imported into a standard word processor (assuming that the user has the sufficient language files installed on their system):\\
\begin{CJK}{UTF8}{min}
\indent{}これは 日本語です。\\
\end{CJK}
\indent{}As you can see, the text is still readable (for those who know how to read the characters in that particular combination), but the ruby text is missing. This can be a problem for novice learners, as they may not know how to read the characters in the given configuration.
\section{Description of Usage}
To use the Ruby package, in order to get Ruby texy, one must first add a reference to the Ruby package (typically in the main source file. In this instance the main source faile is called source.tex).\\
\indent{}This is done by adding the following code to the preamble of the document:
%inform LaTeX to add line numbers to the left of the code, add the numbers to
%each line, and add a frame around the code
\lstset{numbers=left, stepnumber=1, frame=single}
	\begin{lstlisting}
\usepackage{CJKutf8}
\usepackage[overlap]{ruby}
	\end{lstlisting}
\indent{}These two lines of code tell the \LaTeX interpreter to load the files that are required to parse Chinese, Japanese and Korean text that is stored in UTF8 and for processing Ruby text (lines 1 and 2, respecitvely). This is similar to the way that programming languages load class and header files (through an include or using system)\\
\indent{}Once these packages have been added to the preamble of the document, Chinese, Japanese and Korean text can be added with Ruby text overlaying it. To do this, you would add the following code, where neccessary:
\lstset{numbers=left, stepnumber=1, frame=single}
	\begin{lstlisting}
\Ruby{Kanji Character}{Ruby Text}
	\end{lstlisting}
\chapter{Some Source Code}
\epigraph{It's hard enough to find an error in your code when you're looking for it; it's even harder when you've assumed your code is error-free}{Steve McConnell}

%this chapter will contain some source code. It will be pulled in from a file in the same directory
%as the this file (and the rest of the document source files, too)

%the follwoing code is adapted from the code listed at the following url:
% http://texblog.org/2008/04/02/include-source-code-in-latex-with-listings/

\lstset{numbers=left, stepnumber=1}
\lstinputlisting{helloworld.cpp}

%	The following command is the final one that LaTeX will see from the
%	main source file. When it sees this command, it will begin processing
%	the document and producing the output.
%	Everything after this command will be ignored by LaTeX

\end{document}
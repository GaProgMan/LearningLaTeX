Any \LaTeX author can tell from the word go, that the designer of \TeX (which is what \LaTeX is based on) chose a modular paradigm for this system. This choice was deliberate on the designer's part for the simple reason of allowing the author to spend more time actually writing their document rather than editting the layout and look-and-feel of the document.\\
\indent{}As a side note, the original designer (or architect, if you will) of \TeX is Donald Knuth - a champion of excellent algorithm and software design.This author belives that this fact shows when looking at the source code and layout of files for \TeX projects.\\
\indent{}The purpose of this document is to show some of the features that \LaTeX provides. These features will be hand picked by the author as the features that most appeal to him. That is not to say that the author is only interested in a particular subest of features. The author holds the view that when learning a system (beit a language, a toolset or anything else) it is better to focus on the parts of that sytsem which appeal to the learner at first. Then, once the foundation is laid, the learner can then go on to learn other parts of the system.\\
\indent{}The document will be structured such that explanations of the chosen features will be included in the body of text. Further descriptions on the correct usage of the features can be found in the document's soource code.
\begin{center}\emph{N.B: The source code for this document can be obtained from the following website:} \url{http://www.taylorj.org.uk/documents/LaTeX}\end{center}

%a couple of comments about the contents of this file:
% \\ means put a line break here
% \indent{] means indent this paragraph
% \emph{..} means italicise the text ".."

%the code used for justification (the N.B in the above abstract), is adapted from
%the code found at the following url:
%http://www.artofproblemsolving.com/Wiki/index.php/LaTeX:Layout
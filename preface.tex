\chapter*{Preface}
%the above line ensures that the contents of this file are made into a chapter
% (called "Preface"), but that chapter is not added to the table of contents
%(that's what the * is for)

All of the text for the different chapters in this document has been pulled in from files (abstract.tex, preface.tex, chapterOne.tex, etc.) that are seperate to the main file (source.tex).\\
\indent{}This is exceptionally useful, as it means that the author can format the original \LaTeX{} document then keep all of the text that will be added to each section/chapter for said document in seperate files. In other words, if the author wants to change the content of a single section or chapter, then the he  need not worry about the impact on the design for the rest of the document.\\
\indent{}A note on the pronunciation of \TeX and \LaTeX:\\
\indent{}\TeX was originally written, by Donald Knuth, to be used by all writers alike regardless of whether those writers where compiling a technical document, a letter, a manuscript for a book (fact or fiction) or a reference guide for some system or other.\\
\indent{}To reflect this, the characters used to spell \TeX are not those found in the Latin cahracter set, but the Greek character set. Therefore: the word \TeX is spelt with the following Greek letters:\begin{math}\tau, \epsilon and \chi \end{math}(lower cse Tau, Epslion and Chi, repsectively).\\
\indent{}The pronounciation, therefore is based on the Greek alphabet. This means that \TeX is pronounced similar to the (informal) English word "tech" (whcih is short for "Technology").\\
\indent{}Similarly, the name \LaTeX is based on \TeX. This means that it is pronounced either "Lay-Tech" or "Lah-Tech". There is no hard-and-fast rule for the correct  pronouncination of \LaTeX (other than the correct pronounciation of the second part, \TeX).

%the Greek character code and how to add them was adapted from the code found
%at the wollowing url:
%https://en.wikibooks.org/wiki/LaTeX/Mathematics

%the \chapter*{..} code is adapted from an answer to a tex.stackexchange
%question. The answer can be found at the following url:
%http://tex.stackexchange.com/a/19808
\chapter{A Beginning}
\epigraph{A journey of a thousand miles begins with a single step}{Laozi}


%a couple of comments about the contents of this file:
% \chapter{Name of chapter}	- creates a new chapter, on a new page, with the name
%					  "Name of Chapter"
%\ epigrpah{A Quote}{Author}	- adds a quote to the text
%at the minute, the epigraph at the top of the file has been commented out, as it requires
%the epigraph package. This isn't a huge proble, except that it requires all 3gigs of texLive
% be installed (but this will give access to all of the other useful packages)

%the following section of Japanese text is adapted from the tex.stackexchange found at
%the following url: http://tex.stackexchange.com/a/15528

\section{Example of Usage}
\begin{CJK}{UTF8}{min}
これは \ruby{日}{に} \ruby{本}{ほん}\ruby{語}{ご}です。\\
%\\furigana text example: \ruby{山}{やま}\ruby{田}{だ}と\ruby{上}{うえ}\ruby{田}{だ}
\end{CJK}
\indent{}One of the (many) great things about writing documents in \LaTeX is the support for packages.
The above text, when added to a word processor, would produce an output \emph{without} Ruby
text\footnote{see: \url{http://en.wikipedia.org/wiki/Ruby_text} for more details on Ruby text}.
However, due to the modular nature of \LaTeX, adding a single reference to the  package
(\emph{\{ruby\}}) will add the Ruby text (sometimes called \emph{Furigana}) to the document.\\
\indent{}The following
is an example of what the above Japanese characters might look like when imported into a standard
word processor (assuming that the user has the sufficient language files installed on their system):\\
\begin{CJK}{UTF8}{min}
\indent{}これは 日本語です。\\
\end{CJK}
\indent{}As you can see, the text is still readable (for those who know how to read the characters in
that particular combination), but the ruby text is missing. This can be a problem for novice learners, as
they may not know how to read the characters in the given configuration.
\section{Description of Usage}